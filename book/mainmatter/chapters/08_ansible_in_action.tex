\chapter{Ansible in Action: Practical Use Cases}

The theory is great, but let's be honest--seeing Ansible in action is where the magic happens. In this chapter, we'll roll up our sleeves and tackle three real-world scenarios:
\begin{itemize}
    \item Deploying a simple web server (like Nginx or Apache).
    \item Configuring a database server.
    \item Orchestrating multiple services to work together seamlessly.
\end{itemize}

By the end of this chapter, you'll have practical playbooks you can use, adapt, and expand for your own needs. Let's get to it.


\section{Deploying a Simple Web Server}

Let's start with something straightforward: deploying a web server. Whether you're serving a static site or acting as a reverse proxy, tools like Nginx and Apache are staples in any sysadmin's toolkit.

\subsection{Setting Up Nginx}

Here's a playbook to install and configure Nginx on a group of web servers:
\begin{lstlisting}[language=yaml, caption=Playbook for Nginx Deployment]
- name: Deploy Nginx web server
  hosts: webservers
  become: true

  tasks:
    - name: Install Nginx
      ansible.builtin.package:
        name: nginx
        state: present

    - name: Start and enable Nginx
      ansible.builtin.service:
        name: nginx
        state: started
        enabled: yes

    - name: Deploy Nginx configuration
      ansible.builtin.  template:
        src: nginx.conf.j2
        dest: /etc/nginx/nginx.conf

    - name: Restart Nginx to apply changes
      ansible.builtin.service:
        name: nginx
        state: restarted
\end{lstlisting}

\textbf{A} few key things are happening here:
\begin{itemize}
    \item The \texttt{apt} module ensures Nginx is installed.
    \item The \texttt{service} module makes sure it's running and enabled at boot.
    \item The \texttt{template} module deploys a custom configuration file.
    \item Finally, Nginx is restarted to apply any changes.
\end{itemize}

\subsection{Customizing the Configuration}

Here's an example \texttt{nginx.conf.j2} template:
\begin{lstlisting}[language=nginx, caption=Custom Nginx Configuration Template]
server {
    listen 80;
    server_name {{ ansible_hostname }};
    root /var/www/html;

    location / {
        try_files $uri $uri/ =404;
    }
}
\end{lstlisting}

\textbf{C}onfigure it once, and it works across all your web servers. Ansible replaces \texttt{\{\{ ansible\_hostname \}\}} with each server's hostname dynamically.


\section{Configuring a Database Server}

Databases are the backbone of many applications. Let's configure a MySQL server to handle incoming requests reliably.

\subsection{Installing and Securing MySQL}

Here's a playbook to install and secure MySQL:
\begin{lstlisting}[language=yaml, caption=Playbook for MySQL Server Configuration]
- name: Configure MySQL server
  hosts: dbservers
  become: true

  tasks:
    - name: Install MySQL
      ansible.builtin.package:
        name: mysql-server
        state: present

    - name: Start and enable MySQL
      ansible.builtin.service:
        name: mysql
        state: started
        enabled: yes

    - name: Secure MySQL installation
      ansible.builtin.command: >
        mysql_secure_installation
        --use-default
\end{lstlisting}

\textbf{T}his setup ensures MySQL is installed, running, and secure. You can enhance it further by using Ansible modules to manage users and databases.

\subsection{Creating a Database and User}

Let's create a database and a user:
\begin{lstlisting}[language=yaml, caption=Create Database and User]
- name: Create database and user
  hosts: dbservers
  become: true

  tasks:
    - name: Create database
      mysql_db:
        name: myapp
        state: present

    - name: Create database user
      mysql_user:
        name: appuser
        password: secretpassword
        priv: 'myapp.*:ALL'
        state: present
\end{lstlisting}

\textbf{I}n just a few lines, you've set up a database, created a user, and assigned permissions. No manual SQL commands needed.


\section{Orchestrating Multiple Services}

Now for the fun part: making everything work together. Let's deploy a web application that uses Nginx as the frontend and MySQL as the backend.

\subsection{The Playbook}

Here's how you can orchestrate multiple services in a single playbook:
\begin{lstlisting}[language=yaml, caption=Orchestrating Web and Database Services]
- name: Deploy web application
  hosts: all
  become: true

  tasks:
    - name: Install required packages
      ansible.builtin.package:
        name:
          - nginx
          - mysql-server
        state: present

    - name: Configure Nginx
      ansible.builtin.template:
        src: nginx.conf.j2
        dest: /etc/nginx/nginx.conf

    - name: Create application database
      mysql_db:
        name: myapp
        state: present

    - name: Create database user
      mysql_user:
        name: appuser
        password: secretpassword
        priv: 'myapp.*:ALL'
        state: present

    - name: Start services
      ansible.builtin.service:
        name: "{{ item }}"
        state: started
        enabled: yes
      with_items:
        - nginx
        - mysql
\end{lstlisting}

\subsection{Breaking It Down}

\textbf{O}ne playbook, two services:
\begin{itemize}
    \item The \texttt{apt} module installs both Nginx and MySQL.
    \item The \texttt{template} module deploys the Nginx configuration.
    \item The \texttt{mysql\_db} and \texttt{mysql\_user} modules handle database setup.
    \item Finally, the \texttt{with\_items} loop ensures both services are started and enabled.
\end{itemize}

\textbf{N}ow you've got a fully functional web application with a database backend, deployed automatically. It's efficient, repeatable, and scalable.


\section{Wrapping Up}

You've just seen Ansible in action, tackling real-world problems like deploying web servers, configuring databases, and orchestrating services. These examples are just the beginning. The more you use Ansible, the more you'll discover its potential.

\vspace{1em}
\textit{In the next chapter, we'll explore dynamic inventories to make managing complex environments even easier. Ready to dive deeper? Let's go.}
